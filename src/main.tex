\documentclass{abntpuc}
\usepackage[pdftex]{graphicx}
\usepackage{fancyhdr}
\usepackage{booktabs}
\usepackage{graphicx}
\usepackage[table,xcdraw]{xcolor}
\usepackage{lipsum}
\setlength{\headheight}{15pt}
\font\cabecalho=cmr12 at 9pt %-- Opcional para aumentar o tamanho do titulo na capa.
\pagestyle{fancy}
\fancyhf{}
\fancyhead[L]{\emph{\cabecalho\nouppercase\leftmark}}
\fancyhead[R]{\cabecalho\thepage}

\fancypagestyle{plain}{
\fancyhf{}
\renewcommand{\headrulewidth}{0pt}
\fancyhead[R]{\cabecalho\thepage}
}
\begin{document}
\universidade   {Pontifícia Universidade Católica de São Paulo}
\faculdade      {Faculdade de Ciências Exatas}
\departamento   {Departamento de Computação}
\curso          {Ciência da Computação}
\grau           {Bacharel}
\autor          {João E. L. Fouyer}
\trabalho       {Modelo de Trabalho de Conclusão de Curso}
\titulo         {Modelo de Trabalho de Conclusão de Curso}
\cidade         {São Paulo}
\orientador     {Prof. Dr. Donald Ervin Knuth}
\avaliadorB     {Prof. Dr. Edith Ranzini}
\avaliadorA     {Prof. Dr. Paulo Feofiloff}
\ano            {2019}
\datacompleta{22 de maio de 2019}

\palavraschave  {Metadocumento}{Documentação}{ABNT}{Trabalho de Conclusão de Curso}

\preambulo      {Modelo de Trabalho de Conclusão de Curso apresentado à \imprimiruniversidade, como um dos requisitos para a conclusão do Curso de Bacharelado de Ciência da Computação.}

\capa

\folharosto

\begin{center}
\begin{tabular}{|m{0.2cm}p{11.6cm}m{0.2cm}|} \hline
  \hspace{0.3cm} & & \\
  \hspace{0.2cm}  & \hspace{0.3cm} \imprimirtitulo \ / \imprimirautor \ -- \imprimircidade, \imprimirano. & \\
  & \hspace{0.65cm} \pageref{LastPage}f.: il.; 30 cm & \\
  & \hspace{0.4cm} & \\
  & \hspace{0.6cm} \imprimirpreambulo  & \\
  & \hspace{0.6cm} \imprimirorientador & \\
  & & \\
  & \hspace{0.6cm} \imprimirchaves. & \\
  & & \\
  & \hspace {0.6cm}		LINHA DE PESQUISA: &\\  
  & \hspace {0.6cm}		\textbf{Grande Área:} 1.00.00.00-3 Ciências Exatas e da Terra &\\
  & \hspace {0.6cm}	 \textbf{Área:} 1.03.00.00-7 Ciência da Computação&\\
  & \hspace {0.6cm}		\textbf{Subárea:} 1.03.04.00-2 Sistemas de Computação& \\
  & \hspace{4.75cm} & \\
  \hline
\end{tabular}

\end{center}


\folhaaprovacao

\listafiguras
\clearpage
% \listatabelas

% \listasiglas {
%     \sigla{S1}{Sigla 1}
%     \sigla{S2}{Sigla 2}
%     \sigla{S3}{Sigla 3}
% }
\sumario
\chapter{Introdução} \par
O presente documento tem por objetivo servir de modelo para a elaboração de trabalhos acadêmicos --  como monografias, trabalhos de conclusão de curso, teses de metrado etc. A estrutura textual segue o modelo indicado para a elaboração de trabalhos de conclusão para o curso de Ciência da Computação da PUC-SP. 

\section{Motivação}
Deverá ser descrito nesta seção o problema ou a proposta da pesquisa. Geralmente tem de uma a três páginas e não contém os objetivos.

\section{Objetivos}
Nesta seção, os objetivos gerais e específicos deverão ser descritos.

\subsection{Objetivo Geral}
Deve sempre começar no infinitivo.

\subsection{Objetivos Específicos}
Deve sempre começar no infinitivo.

\section{Justificativa}
Pode ser opcional no TCC para trabalhos de conclusão de curso na área da computação.

\section{Delimitação do Problema}
Até onde pesquisar?

\section{Contribuições}
Opcional para trabalhos de conclusão de curso na área da computação.

\section{Método de Pesquisa}
\label{metodo-pesquisa}
Nesta seção, deve-se descrever as macroatividades utilizando-se que quando finalizadas atinge um objetivo. Normalmente varia de seis a doze que podem ser descritas em cascatas. 

\section{Organização do Texto}
Este documento é apresentado em \ref{chap:last} capítulos, referências bibliográficas e os anexos.

No primeiro capítulo, Introdução, descreve-se a problemática, motivação, objetivos gerais e específicos e a metodologia de pesquisa que será utilizada no projeto. 

No capítulo 2, Revisão Bibliográfica, é apresentada a fundamentação teórica que abordará com detalhes a complexidade dos problemas. No caso de desenvolvimento de algum protótipo, descreve-se o método que será utilizado. Por fim, deve-se realizar resumo de pelo menos três trabalhos relacionados à pesquisa.

Por fim, no capítulo \ref{chap:last}, será exposta a análise crítica do problema e quais são possíveis pontos de melhorias para trabalhos futuros.

\newpage
\section{Cronograma}
O cronograma é apresentado com a distribuição das macroatividades propostas no Método de Pesquisa \ref{metodo-pesquisa}.

% Parece difícil criar uma tabela para o cronograma? Não se preocupe! Utilize o gerador de tabelas LaTeX do https://www.tablesgenerator.com/ e gere o código automaticamente.
% Se por algum motivo a tabela não ficar centralizada ou na posição do texto que você quiser, utilize o 'posicionador' H. <https://www.overleaf.com/learn/latex/Positioning_images_and_tables>

\begin{table}[h!]
\centering
\caption{Representação do cronograma das Atividades}
\resizebox{\textwidth}{!}{%
\begin{tabular}{|l|c|c|c|c|
>{\columncolor[HTML]{EFEFEF}}c |c|c|c|c|
>{\columncolor[HTML]{EFEFEF}}c |}
\hline
\textbf{Atividades} & \textbf{fev} & \textbf{mar} & \textbf{abr} & \textbf{mai} & {\color[HTML]{333333} \textbf{jun}} & \textbf{jul} & \textbf{ago} & \textbf{set} & \textbf{out} & \textbf{nov} \\ \hline

Atividade 1 & x & x & x &  & {\color[HTML]{333333} } &  &  &  &  &  \\ \hline

Atividade 2 & x & x & x & x & {\color[HTML]{333333} x} &  &  &  &  &  \\ \hline

Atividade 3 &  &  &  & x & {\color[HTML]{333333} x} & x &  &  &  &  \\ \hline

Atividade 4 &  &  &  & x & {\color[HTML]{333333} x} & x &  &  &  &  \\ \hline

Atividade 5 &  &  &  &  & {\color[HTML]{333333} } & x & x & x & x & x \\ \hline

Atividade 6 &  &  &  &  & {\color[HTML]{333333} } & x & x & x & x & x \\ \hline

Atividade 7 &  &  &  &  & {\color[HTML]{333333} } &  &  & x & x & x \\ \hline

\end{tabular}%
}
\\[10pt]
\caption*{As colunas em destaque representam os \emph{milestones} da defesa do TCC1 e TCC2. Não é necessário apresentar o cronograma na entrega final.}
\end{table}

\chapter{REVISÃO BIBLIOGRÁFICA}\par
\lipsum[1]
\section{Fundamentação Teórica}
Você pode realizar citações diretas utilizando a \emph{chave} presente na bibliografia (\emph{referencias.bib}), como por exemplo \cite{turing}. Ou você pode inserir citações:

\begin{quote}
    The process of preparing programs for a digital computer is especially attractive, not only because it can be economically and scientifically rewarding, but also because it can be an aesthetic experience much like composing poetry or music.
    
    \rightline{{\rm --- \cite[p. 7]{knuth}}} 
\end{quote}

\section{Método de Desenvolvimento}
\lipsum[1]
\section{Trabalhos Relacionados}
\lipsum[1]
\chapter{Desenvolvimento}\par
\lipsum[1]
\section{Donec Varius}
\lipsum[1-2]
\section{Nulla ullamcorper}
\lipsum[1-2]
\chapter{Conclusão}\par
Os objetivos foram atingidos? Quais foram as dificuldades encontradas? E quais foram as soluções para resolver as dificuldades? Qual é a continuidade da pesquisa?

\label{chap:last} % Utilizado para indicar qual é o último capítulo.
\nocite{*}
\renewcommand\bibname{REFERÊNCIAS}
\bibliography{main/referencias}

\newpage
\centering GLOSSÁRIO
\addcontentsline{toc}{chapter}{GLOSSÁRIO}

\newpage
\centering ANEXO
\addcontentsline{toc}{chapter}{ANEXO}

\newpage
\centering APÊNDICE
\addcontentsline{toc}{chapter}{APÊNDICE}

\end{document}

