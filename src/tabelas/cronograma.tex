% Parece difícil criar uma tabela para o cronograma? Não se preocupe! Utilize o gerador de tabelas LaTeX do https://www.tablesgenerator.com/ e gere o código automaticamente.
% Se por algum motivo a tabela não ficar centralizada ou na posição do texto que você quiser, utilize o 'posicionador' H. <https://www.overleaf.com/learn/latex/Positioning_images_and_tables>

\begin{table}[!ht]
\centering
\caption{Representação do cronograma das Atividades}
\resizebox{\textwidth}{!}{%
\begin{tabular}{|l|c|c|c|c|
>{\columncolor[HTML]{EFEFEF}}c |c|c|c|c|
>{\columncolor[HTML]{EFEFEF}}c |}
\hline
\textbf{Atividades} & \textbf{fev} & \textbf{mar} & \textbf{abr} & \textbf{mai} & {\color[HTML]{333333} \textbf{jun}} & \textbf{jul} & \textbf{ago} & \textbf{set} & \textbf{out} & \textbf{nov} \\ \hline

Atividade 1 & x & x & x &  & {\color[HTML]{333333} } &  &  &  &  &  \\ \hline

Atividade 2 & x & x & x & x & {\color[HTML]{333333} x} &  &  &  &  &  \\ \hline

Atividade 3 &  &  &  & x & {\color[HTML]{333333} x} & x &  &  &  &  \\ \hline

Atividade 4 &  &  &  & x & {\color[HTML]{333333} x} & x &  &  &  &  \\ \hline

Atividade 5 &  &  &  &  & {\color[HTML]{333333} } & x & x & x & x & x \\ \hline

Atividade 6 &  &  &  &  & {\color[HTML]{333333} } & x & x & x & x & x \\ \hline

Atividade 7 &  &  &  &  & {\color[HTML]{333333} } &  &  & x & x & x \\ \hline

\end{tabular}%
}
\\[10pt]
\caption*{As colunas em destaque representam os \emph{milestones} da defesa do TCC1 e TCC2. Não é necessário apresentar o cronograma na entrega final.}
\end{table}