\chapter{INTRODUÇÃO}
O presente documento tem por objetivo servir de modelo para a elaboração de trabalhos acadêmicos --  como monografias, trabalhos de conclusão de curso, teses de metrado etc. A estrutura textual segue o modelo indicado para a elaboração de trabalhos de conclusão para o curso de Ciência da Computação da PUC-SP. 

\section{Motivação}
Deverá ser descrito nesta seção o problema ou a proposta da pesquisa. Geralmente tem de uma a três páginas e não contém os objetivos.

\subsection{Subseção}
Subseções podem ser utilizadas para auxiliar na organização do texto.

\subsubsection{Sub-subseção}
Até mesmo sub-subseções também podem ser utilizadas.

\section{Objetivos}
Nesta seção, os objetivos gerais e específicos deverão ser descritos.

\subsection{Objetivo Geral}
Deve descrever de forma abrangente o objetivo geral do trabalho.

\subsection{Objetivos Específicos}
Pode ser escrito em forma de \emph{bullets}. Cada \emph{bullet} deve começar no infinitivo. Para listas não ordenadas em \LaTeX, basta utilizar o comando \emph{itemize}. 

\begin{itemize}
    \item Apresentar um modelo de Trabalho de Conclusão de curso para alunos de computação;
    \item Introduzir os principais elementos do \LaTeX.
\end{itemize}

Para listas enumeradas, é possível utilizar o comando \emph{enumerate}. Caso seja necessário, é permitido encadear listas.

\begin{enumerate}
    \item Item 1
    \begin{itemize}
        \item Sub-item
        \item Sub-item
    \end{itemize}
    \item Item 2
    \begin{enumerate}
        \item Item 2a
        \item Item 2b
    \end{enumerate}
\end{enumerate}

\section{Justificativa}
É oopcional no TCC para trabalhos de conclusão de curso na área da computação.

\section{Delimitação do Problema}
Até onde pesquisar?

\section{Contribuições}
Opcional para trabalhos de conclusão de curso na área da computação.

\section{Método de Pesquisa}
\label{metodo-pesquisa}
Nesta seção, deve-se descrever as macroatividades utilizando-se que quando finalizadas atinge um objetivo. Normalmente varia de seis a doze que podem ser descritas em cascatas. 

\section{Organização do Texto}
Este documento é apresentado em \ref{chap:last} capítulos, referências bibliográficas e os anexos.

No primeiro capítulo, Introdução, descreve-se a problemática, motivação, objetivos gerais e específicos e a metodologia de pesquisa que será utilizada no projeto. 

No capítulo 2, Revisão Bibliográfica, é apresentada a fundamentação teórica que abordará com detalhes a complexidade dos problemas. No caso de desenvolvimento de algum protótipo, descreve-se o método que será utilizado. Deve-se realizar resumo de pelo menos três trabalhos relacionados à pesquisa.

Por fim, no capítulo \ref{conclusao}, será exposta a análise crítica do problema e quais são possíveis pontos de melhorias para trabalhos futuros.

\newpage
\section{Cronograma}
O cronograma é apresentado com a distribuição das macroatividades propostas no Método de Pesquisa \ref{metodo-pesquisa}. O \LaTeX Table Generator pode ser utilizado para auxiliar na elaboração de tabelas. (https://www.tablesgenerator.com/). 

% Parece difícil criar uma tabela para o cronograma? Não se preocupe! Utilize o gerador de tabelas LaTeX do https://www.tablesgenerator.com/ e gere o código automaticamente.
% Se por algum motivo a tabela não ficar centralizada ou na posição do texto que você quiser, utilize o 'posicionador' H. <https://www.overleaf.com/learn/latex/Positioning_images_and_tables>

\begin{table}[h!]
\centering
\caption{Representação do cronograma das Atividades}
\resizebox{\textwidth}{!}{%
\begin{tabular}{|l|c|c|c|c|
>{\columncolor[HTML]{EFEFEF}}c |c|c|c|c|
>{\columncolor[HTML]{EFEFEF}}c |}
\hline
\textbf{Atividades} & \textbf{fev} & \textbf{mar} & \textbf{abr} & \textbf{mai} & {\color[HTML]{333333} \textbf{jun}} & \textbf{jul} & \textbf{ago} & \textbf{set} & \textbf{out} & \textbf{nov} \\ \hline

Atividade 1 & x & x & x &  & {\color[HTML]{333333} } &  &  &  &  &  \\ \hline

Atividade 2 & x & x & x & x & {\color[HTML]{333333} x} &  &  &  &  &  \\ \hline

Atividade 3 &  &  &  & x & {\color[HTML]{333333} x} & x &  &  &  &  \\ \hline

Atividade 4 &  &  &  & x & {\color[HTML]{333333} x} & x &  &  &  &  \\ \hline

Atividade 5 &  &  &  &  & {\color[HTML]{333333} } & x & x & x & x & x \\ \hline

Atividade 6 &  &  &  &  & {\color[HTML]{333333} } & x & x & x & x & x \\ \hline

Atividade 7 &  &  &  &  & {\color[HTML]{333333} } &  &  & x & x & x \\ \hline

\end{tabular}%
}
\\[10pt]
\caption*{As colunas em destaque representam os \emph{milestones} da defesa do TCC1 e TCC2. Não é necessário apresentar o cronograma na entrega final.}
\end{table}
