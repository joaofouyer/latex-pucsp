\documentclass{abntpuc}

\universidade   {Pontifícia Universidade Católica de São Paulo}
\faculdade      {Faculdade de Ciências Exatas}
\departamento   {Departamento de Computação}
\curso          {Ciência da Computação}
\grau           {Bacharel}
\author         {João E. L. Fouyer}
\trabalho       {Relatório Final de Projeto de Iniciação Científica}
\titulo         {Modelo de Projeto de Pesquisa para Iniciação Científica}
\cidade         {São Paulo}
\orientador     {Prof. Dr. Alan Turing}
\avaliadorA     {Profa. Lisbete Madsen Barbosa}
\avaliadorB     {Prof. Julio Arakaki}
\ano            {2019}
\datacompleta   {22 de maio de 2019}

\palavraschave  {Modelo ABNT}{PUC-SP}{Iniciação Científica}{Relatório Final}

\preambulo      {Modelo de Projeto de Pesquisa para Iniciação Científica da Pontifícia Universidade Católica de São Paulo.}


\begin{document}

\capa

\folharosto

\sumario

\chapter{INTRODUÇÃO} 

Desenvolva de modo geral e abrangente a relevância, motivação, justificativa e a problemática daquilo que será pesquisado.

\section{PROBLEMA DE PESQUISA}

Geralmente é exposta na forma de uma pergunta que a pesquisa será pautada em respondê-la.

\section{OBJETIVOS}

Introduz de modo abrangente os objetivos da pesquisa.

\subsection{OBJETIVOS GERAIS}

Descreve sucintamente o objetivo geral da pesquisa.

\subsection{OBJETIVOS ESPECÍFICOS}

Lista os objetivos específicos da pesquisa. Pode ser feito através de bullets. Cada bullet deve começar com um verbo no infinitivo.

\begin{itemize}
    \item Objetivo Específico 1;
    \item Objetivo Específico 2;
    \item Objetivo Específico N.
\end{itemize}

\section{REFERENCIAL TEÓRICO}

Introduza as referências bibliográficas que serão utilizadas no desenvolvimento da sua pesquisa. Pode incluir citações, como por exemplo:

\begin{quote}
    "Aquilo que pode ser afirmado sem provas, pode também ser negado sem provas."
\end{quote}
\rightline{\cite{euclides}}

\section{METODOLOGIA DE PESQUISA}

Nesta seção, deve ser descrito como a pesquisa será realizada.

\section{CRONOGRAMA}

O cronograma deve apresentar as macroatividades -- como a elaboração do relatório parcial e final --  da pesquisa e sua estimativa de prazo. 

\begin{table}[!ht]
\centering
\caption{Representação do cronograma das Atividades}
\resizebox{\textwidth}{!}{%
\begin{tabular}{|l|c|c|c|c|c|c|c|c|c|c|c|c|}
\hline
\textbf{Atividades} & \textbf{ago} & \textbf{set} & \textbf{out} & \textbf{nov} & {\color[HTML]{333333} \textbf{dez}} & \textbf{jan} & \textbf{fev} & \textbf{mar} & \textbf{abr} & \textbf{mai} & \textbf{jun} & \textbf{jul} \\ \hline

Atividade 1 & x & x & x &  &  & & {\color[HTML]{333333} } &  &  &  &  &  \\ \hline

Atividade 2 & x & x & x & x & {\color[HTML]{333333} x} &  &  &  &  &  & & \\ \hline

Atividade 3 &  &  &  & x & {\color[HTML]{333333} x} & x &  &  &  &  & & \\ \hline

Atividade 4 &  &  &  & x & {\color[HTML]{333333} x} & x &  &  &  &  & & \\ \hline

Atividade 5 &  &  &  &  & {\color[HTML]{333333} } & x & x & x & x & x  & & \\ \hline

Atividade 6 &  &  &  &  & {\color[HTML]{333333} } & & x & x & x & x  & x & \\ \hline

Atividade 7 &  &  &  &  & {\color[HTML]{333333} } &  &  &  &  & x  & x & x\\ \hline

\end{tabular}%
}
\\[10pt]
\end{table}


\renewcommand\bibname{REFERÊNCIAS}
\addcontentsline{toc}{chapter}{\bibname}
\bibliography{referencias}


\end{document}

