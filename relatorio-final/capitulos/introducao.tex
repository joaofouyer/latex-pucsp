\chapter{Introdução}

O relatório final é elaborado como forma de expor os resultados obtidos ao decorrer do desenvolvimento de acordo com o projeto de pesquisa da Iniciação Científica. Deve conter a análise da metodologia utilizada pelo aluno e pelo orientador para demonstração ao Comitê Institucional. O relatório final é previsto para ser entregue doze meses após a vigência do período de aprovação, apresentando análises conclusivas dos objetivos gerais e específicos do projeto de pesquisa aprovado.

É recomendado que este documento seja dividido entre duas partes principais, onde a primeira parte apresenta relatório das atividades e expõe uma análise sobre a sistemática da orientação e descrição dos objetivos gerais e específicos. Já a segunda parte consiste em um relatório científico, que contém resultados finais da pesquisa, discussão crítica dos resultados e análise final do projeto.