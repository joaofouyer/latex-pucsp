\newpage
\begin{center}
    RESUMO
\end{center}
2.00.00.00-6 – CIÊNCIAS BIOLÓGICAS 
\\
2.08.00.00-2 – BIOQUÍMICA
\\
ANÁLISE   DA   EXPRESSÃO   DAS   PROTEÍNAS  NA HIPERTROFIA  MUSCULAR  ESQUELÉTICA SUPLEMENTADOS COM LEUCINA NA DIETA 
\\
GISLAINE VENTRUCCI – ORIENTADORA
\\
Departamento de Ciências Fisiológicas –Faculdade de Ciências Médicas e da Saúde
\\
gventrucci@pucsp.br
\\
DÉBORAH MENDES SOARES – ORIENTANDA 
\\
Curso de Medicina -Faculdade de Ciências Médicas e da Saúde
\\
deborah.estudos@outlook.com
\\

A  sarcopenia  é  uma  alteração  do  envelhecimento  e  refere-se  à  perda  progressiva  e generalizada  da  massa  muscular  esquelética.  Sugere-se  que  essa  é uma  condição multifatorial  que  pode  ser  consequência  de  nutrição  inadequada  e  inatividade  física. Avaliou-se o padrão de expressão dos fatores envolvidos na síntese proteica do músculo esquelético  (eIF4G  e  S6K1)  de  animais  idosos  submetidos  ao  exercício  físico  e  a suplementação   nutricional   com   leucina.   O   aumento   da   expectativa   de   vida   torna necessário estudos sobre a sarcopenia e a associação entre atividade física e uma dieta rica  em  leucina  é  uma  alternativa  promissora.  Este  é  continuação  de  um  projeto  com  35 ratos  Wistar  adultos.  Foram  utilizadas  a  dieta  controle  normoproteica  e  dieta  rica  em leucina.  Foram  formados  5  grupos:  adultos  controle  com  dieta  controle  (A);  idosos sedentários  tratados  dieta  controle  (IS);  idosos  treinados  tratados  dieta  controle(IT); idosos  sedentários  tratados  com  dieta  rica  em  leucina  (LS)  e  idosos  treinados  tratados com dieta rica em leucina (LT). Após o sacrifício, realizou-se a quantificação de eIF4G e S6K1 com a técnica de western Blotting, sendo a concentração de proteína total avaliada pelo método de Lowry, a quantidade pela eletroforese em gel de poliacrilamida, com o kit de quimioluminescência para detecção das bandas. Para determinação da eIF4G e S6K1 utilizou-se   anticorpos   primário   anti-goatpoliclonal,   fazendo   ligação com   o   anticorpo secundário.   Para   a   quimioluminescência   utilizou-se   reagente   ECL   e   para   análise densitométrica da banda utilizou-se Image Capture (software Gel Pro II). Observou-se que o  grupo  IS  apresentou  redução  do  músculo  gastrocnêmio  comparado  ao  grupo  A.  \textbf{PIBIC-CNPq}
